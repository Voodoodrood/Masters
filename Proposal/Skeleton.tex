\documentclass{sig-alternate-05-2015}
\usepackage{float}
\usepackage{enumitem}
\usepackage{subfiles}
\usepackage[english]{babel}
\usepackage{graphicx}
\usepackage{epstopdf}
\usepackage{tabularx}

\makeatletter
\def\@copyrightspace{\relax}
\makeatother

\begin{document}
\title{A user interface for landscape modelling in a virtual environment using a head mounted display}

\numberofauthors{1}
\author{
\alignauthor
Timothy Gwynn\\
       \affaddr{GWYTIM001}\\
       \affaddr{University of Cape Town}
}
\maketitle
\begin{CCSXML}

\end{CCSXML}



\printccsdesc
\keywords{Virtual Environments, Terrain Synthesis, User Testing}
\begin{abstract}
	
\end{abstract}
\section{Introduction}
Introduce the concept of terrain synthesis and the areas for which it is useful. Talk about the recent advances and availability of VR hardware and why it is important/interseting. Also address the imbalance in hardware/software research and the need for user testing of interface design in VEs

\section{Background Reading}
This is the stuff that is related and necassary for the research but not necassarily similar.
\subsection{Terrain Synthesis}
Discuss papers on terrain synthesis and ways in which interactivity is currently incorporated in it.
\subsection{Interface design for modelling in large-scale Virtual environments}
Large section, can be broken down into navigation, interface interaction and environment interaction. Cover a variety of principles in each section and how they need to be integrated to create a useful interface.
\subsection{Usability testing}
Include papers on user testing particularly with regards to usability. Try use papers that are VE specific but many principles here are generic. Note, discuss the need for telemetry and cable handling.
\section{Related Work}
Previous work that closely resembles the project.
\subsection{Comparison of Desktop applications to Virtual environment}
Talk about previous expirements comparing desktop systems to VE systems. This forms the primary motivation for the research as these show people perform better in VEs. Also highlight the neccesuty for relatively high training times.
\subsection{Designing for Virtual Environments in Virtual Environments}
Other projects that created systems for modelling in VEs. Discuss what worked and what didn't. Also talk about the various devices and control schemes that were implemented and which ones showed promise.
\section{Research question(s)}
Perhaps the most important section. Still need to come up with solid question. Justify why question is good based on points made in PG reseach methods
\section{Method}
Describe the way in which I will carry out the expirement.
\subsection{System Design}
Both hardware and software specifications, based on what worked from related work and principles from background reading. Need to justify design decisions for the system.
\subsection{Experimental design}
Define the exact process of the experiment. This includes the tasks users must perform and what is measured. Also mention aspects such as user training and user selection.
\subsection{Data collection}
Talk about how data will be collected eg. telemetry, think out loud, questionaires etc. Also specify what data will be collected (usability scale, flow scale, speed etc.)
\subsection{Data analysis}
Explain how we plan to analyse the data, what statistical methods will be used etc.
\section{Project plan}
Discuss what the plan is going forward from here, show that the scope is practical within time and resource constraints.
\subsection{Requirements}
List the hardware and software required to complete the project, primarily for system design but also for the testingand data collection.
\subsection{Risks}
Make a table of risks, define their riskiness and give strategies for combatting them.
\subsection{Timeline}
Make a gant chart, show that everything can be done in 18 months
\section{Research outcomes}
What does the universe at large get out of this.
\subsection{Artifacts produced}
Talk about the sofware system that will be created in the process of research.
\subsection{Success factors}
Define the conditions which we need to fulfil in order to conclude that the project has been successful. (Not the same as proving the hypothesis true, need to take into consideration that a negative result can still be a succsessful outcome)
\subsection{Relevance to industry}
What are the implications of this research in industry, particularly the game industry. A certain level of success suggests that companies designing VE products should look into using VE tools during the design process.
\subsection{Further research}\
How can this research be extended. Looking at more general systems and interface design, doing more user testing.

\bibliographystyle{abbrv}
\bibliography{proposal}
\end{document}\textbf{}