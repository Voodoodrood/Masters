\documentclass{sig-alternate-05-2015}
\usepackage{float}
\usepackage{enumitem}
\usepackage{subfiles}
\usepackage[english]{babel}
\usepackage{graphicx}
\usepackage{epstopdf}
\usepackage{tabularx}

\makeatletter
\def\@copyrightspace{\relax}
\makeatother

\begin{document}
\title{Paper Prototype design and feedback}

\numberofauthors{1}
\author{
\alignauthor
Timothy Gwynn\\
       \affaddr{GWYTIM001}\\
       \affaddr{University of Cape Town\\
       Supervisor: J. Gain}
}
\maketitle
\begin{CCSXML}

\end{CCSXML}



\printccsdesc
\section{Paper Prototype}
\subsection{Design}
The paper prototype allowed the user to interact with all UI elements that were planned to be included in the VE interface. It also included a number of user instructions to guide first time uses through the program. Although the interface elements would react to user input it was not feasible to have the terrain model itself to react. Thus users were required to imagine how the terrain would change based on verbal feedback.

The prototype was constructed using a number of parts to represent various interface elements. The displayed terrain itself was represented by a flat sheet of paper on a table surface. This allowed UI elements to be placed onto the terrain surface in approximation to the functionality of the VE. Users were asked to keep in mind that the terrain image was simply a representation of the VR display. A number of UI elements were represented by small cardboard tokens which could be placed on the terrain or "view area". Two ring menus were included, one being the main menu which users could select tools with and the other being a terrain type menu which users
\subsection{Evaluation}
\subsection{Changes to design}


\end{document}\textbf{}